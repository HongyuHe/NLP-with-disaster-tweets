\section{Discussion}
In this report, we demonstrate the results of various machine learning models in the task of natural language classification.
The end goal was to explore potential solutions to the problem of classifying tweets if they pertain to disasters or not.
With some success, we demonstrated that this is possible but not with high levels of confidence.
This lack of success stems from the problem statement being overly broad, in that it is not precisely clear what a disaster is and what is not.
There is not an authoritative source of truth that defines what is and is not a disaster because the meaning of this word is different according to different people.
Rightly so as different people have different perspectives and sets of values.
For example, to Donald Trump, the North American Free Trade Agreement is a disaster.
However, citizens of Mexico would say that it is an overwhelming success that has brought wealth and prosperity to their nation.
So is NAFTA a disaster or not? To Americans that worked in Automotive part suppliers (brakes, engines, etc.) NAFTA was a disaster since it decreased the number of available jobs in the country, but the opposite is true for Mexico and other trading partners of the US.
This begs the question, to whom would such an intelligent system serve? Americans? It could, but it would be unfair.
Therefore the scope of this problem needs to be narrowed down to something more specific that is unquestionably objective.
Without the problem adequately modeled, there is no set of equations that can consistently deliver accurate results for all cases.


One way to improve this system is to target, for example, natural disasters or automotive accidents.
An additional improvement would be to include a knowledge base that allows the system to understand that, for example, Kansas is situated in tornado valley, and so if a tweet becomes geotagged from Kansas, then it is more likely about a disaster.
This knowledge-base could also have a memory, that if there is a known disaster occurring in a geographical area, then tweets from that area will be much more likely to be about a disaster.


If this system were to be designed and implemented, then a thorough discussion about the acceptable level of risk associated with false positives and negatives.
As in how costly would a mistake be from this system trying to classify tweets? It is conceivable that if the risks are low, then it does not warrant a machine learning solution and all the complexity that comes from it.
Along this vein of discussion, perhaps no classifier is the best classifier.
Twitter could conceivably reserve a keyword in their service to flag to some trivial system that a particular tweet is, in fact, about a disaster.
If a user wants to let others know that they are in a disaster, then they can use it.
This system could be abused or misused, but the same holds for the machine learning models.


Regardless of the implementation, this system would be incredibly useful to organizations that dispatch critical resources to regions affected by disasters.
Such a system would allow for a more automated and proactive process that could make the difference in having a memorable disaster or a mundane event.
