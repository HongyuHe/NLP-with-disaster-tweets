\section{Introduction}
In this report, we describe the NLP (Natural Language Processing) pipeline to compete in the ``Real or Not? NLP with Disaster Tweets'' Kaggle competition.
The goal of the competition is to decide, given a tweet, whether it pertains to a disaster or not.
The NLP pipeline with the highest accuracy wins the competition.
The prepared dataset was provided by Figure Eight Inc.
and the pipeline consisted of Python libraries.

\subsection{Hypothesis}
Our research will focus on exploring the difference in the models’ performance depending on the feature set. Similar research has shown the potential
of NLP feature extraction techniques to improve the predictions \cite{y1,y2,y3}.
Thus, our research will focus on investigating the hypothesis: \textit{Linguistic features extracted using the natural language processing pipeline
improve the classification of disaster tweets compared to classic derivable features.}

\subsection{Approach}
In order to investigate our hypothesis, we have decided on a structured approach.
As the first step, we performed pre-processing on the provided training data, which included data analysis and data cleaning.
This step consisted of investigating the class imbalance, dealing with missing values and seeking correlations between features and classes among others.
Based on the latter we have built a set of features which could be potentially useful for the classification.
These will be later used for the training pipeline.
For model evaluation, cross-validation with 5 stratified folds was used.
This ensured that every instance from the dataset has been used for the training at least once.
Moreover, the folds contained approximately the same proportion of the target value as the original dataset.
To form the final conclusions, we evaluated our models on the test data.
The respective steps along with the reasoning are described in-depth in the following sections.

